\documentclass{scrartcl}
\usepackage[a4paper, left=3cm, right=3cm, top=2.4cm]{geometry}
\usepackage[utf8]{inputenc}
\usepackage[german]{babel}
\usepackage{url}
%Code
\usepackage{listings}
\usepackage[table, usenames, dvipsnames]{xcolor}
\definecolor{defcolor}{rgb}{0,0.6,0}
\definecolor{commentcolor}{rgb}{0.6,0.6,0.6}
\definecolor{stringcolor}{rgb}{0.8,0,0}
\lstset{frame=single,
	language=C,
	aboveskip=3mm,
	belowskip=3mm,
	showstringspaces=false,
	columns=flexible,
	basicstyle={\small\ttfamily},
	numbers=none,
	numberstyle=\tiny\color{gray},
	keywordstyle=\color{defcolor},
	commentstyle=\color{commentcolor},
	stringstyle=\color{stringcolor},	
	breaklines=false,
	breakatwhitespace=false,
	tabsize=3,
	captionpos=b
}
%Images
\usepackage{graphicx}
\graphicspath{ {images/} }
%biblatex
\usepackage[
backend=bibtex,
style=verbose,
maxbibnames=99,
]{biblatex}
\addbibresource{Ausarbeitung.bib}
%URLs
\definecolor{UrlColor}{HTML}{155196}    % url color
\definecolor{CiteColor}{HTML}{6200d9}	% cite color (footnotes)
\definecolor{TocColor}{HTML}{003F61}	% toc color
\usepackage[%
colorlinks={true},
linkcolor = TocColor,
urlcolor = UrlColor,
citecolor = CiteColor,
%breaklinks=true,
]{hyperref}


\begin{document}
{
	\centering
	\includegraphics[width=\textwidth]{logo.png}
	\ \\\ \\
}
\begin{Form}
	{
		\huge
		\centering
		\textbf{Beitrittserklärung}\\\ \\
	}
	\noindent
	\begin{tabularx}{\linewidth}{p{9cm} p{9cm}}
		\TextField[name=nachname, width=7cm, bordercolor={black}, borderstyle=U, backgroundcolor=lightergray]{Nachname:\hfill} &
		\TextField[name=vorname, width=7.7cm, bordercolor={black}, borderstyle=U, backgroundcolor=lightergray]{Vorname:\hfill}\\\\
		\TextField[name=strasse, width=7cm, bordercolor={black}, borderstyle=U, backgroundcolor=lightergray]{Straße:\hfill} &
		\TextField[name=plzort, width=7.7cm, bordercolor={black}, borderstyle=U, backgroundcolor=lightergray]{PLZ, Ort:\hfill}\\\\
	\end{tabularx}
	\begin{tabularx}{\linewidth}{p{5cm} p{5.5cm} p{7.5cm}}
		\TextField[name=geb, width=3cm, bordercolor={black}, borderstyle=U, backgroundcolor=lightergray]{Geb.am:\hfill} &
		\TextField[name=tel, width=4cm, bordercolor={black}, borderstyle=U, backgroundcolor=lightergray]{Telefon:\hfill} &
		\TextField[name=mail, width=6cm, bordercolor={black}, borderstyle=U, backgroundcolor=lightergray]{E-Mail:\hfill}\\\\
	\end{tabularx}\\\\
	Ich erkläre hiermit meinen Beitritt zum \textbf{Tennisclub Biberach e.V.} als Mitglied und erkenne die Vereinssatzung an.	Die Mitgliedschaft wird gewünscht als (eines ankreuzen):\\\\
	% TODO \Use ChoiceMenu when hyperref bug is fixed (https://github.com/ho-tex/hyperref/issues/3)
	% set boxes on the left (https://tex.stackexchange.com/questions/446934/latex-forms-place-label-on-the-right-side-of-checkbox)
	\def\LayoutCheckField#1#2{% label, field
		#2 #1%
	}
	\begin{tabularx}{\linewidth}{p{5cm} p{5cm} p{8cm}}
		\CheckBox[name=kinder, bordercolor={black}]{Kinder (bis 13 J.)} &
		\CheckBox[name=jugend, bordercolor={black}]{Jugendliche (14 J. - 17 J.)} &
		\CheckBox[name=azubi, bordercolor={black}]{Azubis, Schüler, Studenten (ab 18 J.)}\\\\
		\CheckBox[name=einzel, bordercolor={black}]{Einzelmitglieder (ab 18 J.)} &
		\CheckBox[name=zweit, bordercolor={black}]{Zweitmitgliedschaft} &
		\CheckBox[name=familie, bordercolor={black}]{Familienmitgliedschaft (inkl. Kinder bis 17 J.)}\\\\
		\CheckBox[name=passiv, bordercolor={black}]{Passiv} &
		\multicolumn{2}{p{13cm}}{\CheckBox[name=schnupper, bordercolor={black}]{Schnuppermitgliedschaft (für laufendes Kalenderjahr, incl. 2 Trainerstunden)}} \\
	\end{tabularx}
	\ \\\\
	\begin{tabularx}{\linewidth}{p{9cm} p{9cm}}
		\TextField[name=ortdatuma, width=7cm, bordercolor={black}, borderstyle=U, backgroundcolor=lightergray]{Ort, Datum:\hfill} &
		Unterschrift: \xrfill[-5pt]{0.5pt}\\\\
		\multicolumn{2}{p{18.4cm}}{Unterschrift gesetzlicher Vertreter (bei Jugendlichen unter 18 Jahren): \xrfill[-5pt]{0.5pt}}
	\end{tabularx}
	\ \\\\{\centering \rule{18.6cm}{2pt}\vspace{0.3cm}}\\
	\textbf{Ermächtigung zum Einzug von Forderungen mittels SEPA-Lastschriftsmandat}\\
	Ich ermächtige den TC Biberach e.V. (Gläubiger ID DE17ZZZ00000301236), Zahlungen von meinem Konto mittels Lastschrift einzuziehen. Zugleich weise ich mein Kreditinstitut an, die von TC Biberach e.V. auf mein Konto gezogenen Lastschriften einzulösen.\\
	Hinweis: Ich kann innerhalb von acht Wochen, beginnend mit dem Belastungsdatum, die Erstattung des belasteten Betrages verlangen. Es gelten dabei die mit meinem Kreditinstitut vereinbarten Bedingungen.\\\\
	\begin{tabularx}{\linewidth}{p{9cm} p{9cm}}
		\TextField[name=kontoinhaber, width=6cm, bordercolor={black}, borderstyle=U, backgroundcolor=lightergray]{Kontoinhaber:\hfill} &
		\TextField[name=bank, width=7.5cm, bordercolor={black}, borderstyle=U, backgroundcolor=lightergray]{Bank:\hfill}\\\\
		\TextField[name=iban, width=7cm, bordercolor={black}, borderstyle=U, backgroundcolor=lightergray]{IBAN:\hfill} &
		\TextField[name=bic, width=7.5cm, bordercolor={black}, borderstyle=U, backgroundcolor=lightergray]{BIC:\hfill} \\\\\\
		\TextField[name=ortdatumb, width=7cm, bordercolor={black}, borderstyle=U, backgroundcolor=lightergray]{Ort, Datum:\hfill} &
		Unterschrift: \xrfill[-5pt]{0.5pt}\\\\
	\end{tabularx}
	\ \\{\centering \rule{18.6cm}{2pt}\vspace{0.3cm}}\\
	\begin{tabularx}{\linewidth}{p{9cm} p{9cm}}
		\multicolumn{2}{p{18.6cm}}{\CheckBox[name=schluessel, bordercolor={black}]{Schlüssel für Platzanlage erhalten}}\\\\
		\TextField[name=ortdatumc, width=7cm, bordercolor={black}, borderstyle=U, backgroundcolor=lightergray]{Ort, Datum:\hfill} &
		Unterschrift: \xrfill[-5pt]{0.5pt}\\\\
	\end{tabularx}
	\begin{center}
		\footnotesize
		Tennisclub Biberach e.V. - 77781 Biberach / Baden\\
		\url{www.tcbiberach.de} $\bullet$ \href{mailto:info@tcbiberach.de}{info@tcbiberach.de}\\\vspace{0.1cm}
		\begin{tabular}{lll}
			Sparkasse Haslach-Zell &
			IBAN: DE17 6645 1548 0027 0200 07 &
			BIC: SOLADES1HAL\\
			Volksbank Lahr eG &
			IBAN: DE85 6829 0000 0045 0293 01 &
			BIC: GENODE61LAH\\
		\end{tabular}
	\end{center}
\end{Form}
\newpage
{
	\centering
	\includegraphics[width=\textwidth]{logo.png}
	\ \\\ \\
}
\begin{Form}
	{
		\huge
		\centering
		\textbf{Pflichten für Aktivmitglieder}\\
	}
	\noindent
	\begin{itemize}
		\item Eine Woche Clubheimbewirtung im Jahr (ab 18 Jahren, in Gruppen von 3--4 Personen möglich)
		\item 6 Arbeitsstunden im Jahr leisten (ab 16 Jahren)
	\end{itemize}
	{
		\huge
		\centering
		\textbf{Datenschutzerklärung}\\\ \\
	}
	\noindent
	\textbf{\Large Allgemeine Vereinbarung}\\
	Die in der Beitrittserklärung angegebenen personenbezogenen Daten, insbesondere Name, Anschrift, Geburtsdatum, Telefonnummer, E-Mail, Bankdaten, die zum Zwecke der Durchführung des entstehenden Mitgliedsverhältnisses notwendig und erforderlich sind, werden auf Grundlage gesetzlicher Berechtigungen erhoben. Diese Informationen werden in dem vereinseigenen EDV-System gespeichert. Jedem Vereinsmitglied wird dabei eine Mitgliedsnummer zugeordnet. Die personenbezogenen Daten werden durch geeignete technische und organisatorische Maßnahmen vor der Kenntnisnahme Dritter geschützt. Vorstandsmitglieder des Vereins sind im Rahmen geltender Beschlüsse des Vorstandes befugt personenbezogene Daten des Mitglieds ausschließlich und alleine für Vereinszwecke zu verarbeiten. Das Mitglied stimmt dieser Art und Weise der Verarbeitung durch seine Mitgliedschaft im Verein zu.\\\\
	\noindent
	\textbf{\Large Pressearbeit}\\
	Der Verein kann über Print- und Telemedien sowie soziale Medien und auf seiner Homepage \url{www.tcbiberach.de} regelmäßig über Turnierergebnisse und besondere Ereignisse (z.B. Ehrungen) informieren.	Dabei können personenbezogene Daten des Mitglieds wie Name, Geburtsdatum, Spielergebnisse, Beitrittsdatum sowie Fotos, die im Rahmen von Aktivitäten des Tennisclub Biberach e.V entstanden sind, veröffentlicht werden. Das einzelne Mitglied kann jederzeit gegenüber dem Vorstand einer solchen Veröffentlichung widersprechen. Im Falle des Widerspruchs unterbleiben in Bezug auf das widersprechende Mitglied weitere Veröffentlichungen. Personenbezogene Daten des widersprechenden Mitglieds werden von der Homepage des Vereins entfernt.\\\\
	\noindent
	\textbf{\Large Austritt aus dem Verein}\\
	Beim Austritt werden die personenbezogenen Daten des Mitglieds archiviert. Personenbezogene Daten des austretenden Mitglieds, die die Kassenverwaltung betreffen, werden gemäß der steuergesetzlichen Bestimmungen bis zu zehn Jahre ab der schriftlichen Bestätigung des Austritts im vereinseigenen EDV-System gespeichert.\\\\
	\noindent
	\textbf{\Large Betroffenenrechte}\\
	Sie sind jederzeit dazu berechtigt die folgenden Rechte ausüben:
	\begin{itemize}
		\setlength\itemsep{0em} % fix spacing between items
		\item Auskunft über Ihre bei uns gespeicherten Daten und deren Verarbeitung (Art. 15 DSGVO)
		\item Berichtigung unrichtiger personenbezogener Daten (Art. 16 DSGVO)
		\item Löschung Ihrer bei uns gespeicherten Daten (Art. 17 DSGVO)
		\item Einschränkung der Datenverarbeitung, sofern wir Ihre Daten aufgrund gesetzlicher Pflichten noch nicht löschen dürfen (Art. 18 DSGVO)
		\item Datenübertragbarkeit, sofern Sie in die Datenverarbeitung eingewilligt haben oder einen Vertrag mit uns abgeschlossen haben (Art. 20 DSGVO)
		\item Widerspruch gegen die Verarbeitung Ihrer Daten bei uns (Art. 21 DSGVO)
	\end{itemize}
	\ \\\\
	\begin{tabularx}{\linewidth}{p{9cm} p{9cm}}
		\TextField[name=ortdatumd, width=7cm, bordercolor={black}, borderstyle=U, backgroundcolor=lightergray]{Ort, Datum:\hfill} &
		Unterschrift: \xrfill[-5pt]{0.5pt}\\\\
		\multicolumn{2}{p{18.4cm}}{Unterschrift gesetzlicher Vertreter (bei Jugendlichen unter 18 Jahren): \xrfill[-5pt]{0.5pt}}
	\end{tabularx}
\end{Form}
\newpage
{
	\centering
	\includegraphics[width=\textwidth]{logo.png}
	\ \\\ \\
}
{
	\huge
	\centering
	\textbf{Beitrags- und Gebührenordnung}\\\ \\
}
\noindent
\begin{table}[H]
	\centering
	\begin{tabular}[c]{lr}
		\textbf{Mitgliedschaft} & \textbf{Jahresbeitrag}\\\hline
		Kinder bis 13 Jahre & 20,00\texteuro\\
		Jugendliche von 14 - 17 Jahre & 40,00\texteuro\\
		Azubis, Schüler, Studenten ab 18 Jahre & 70,00\texteuro\\
		Einzelmitglieder ab 18 Jahre & 140,00\texteuro\\
		Zweitmitgliedschaft Erwachsene & 1/2 Beitrag\\
		Familienmitgliedschaft (inkl. Kinder bis 17 Jahre) & 215,00\texteuro\\
		Schnuppermitgliedschaft (für laufendes Kalenderjahr, incl. 2 Trainerstunden) & 50,00\texteuro\\
		Passive Mitglieder & 35,00\texteuro\\
		&\\
		\textbf{Gebühren} & \textbf{Jahresbeitrag}\\\hline
		Nichtbewirtung 18-70 Jahre (1 Woche Clubheimbewirtung/Jahr) & 50,00\texteuro\\
		Nicht geleistete Arbeitsstunden 16-70 Jahre (6 Std. x 10,00 \texteuro) & 60,00\texteuro\\
		&\\
		\textbf{Gästekarten} & \textbf{Einmalbeitrag}\\\hline
		Passives Mitglied & 4,00\texteuro\\
		Nichtmitglied & 5,00\texteuro\\
		2 Nichtmitglieder & 10,00\texteuro\\
	\end{tabular}
\end{table}
\ \\
{
	\huge
	\centering
	\textbf{Pflichten}\\\ \\
}
\noindent
Einmal im Jahr ist \textbf{eine Woche Clubheimbewirtung} zu leisten (für Aktivmitglieder ab 18 Jahren).\\
Die Bewirtung kann in Gruppen von 3-4 Personen gemeinsam erbracht werden.\\\\
Im Jahr sind \textbf{6 Arbeitsstunden} zu leisten (ab 16 Jahren).\\\\
Bei Nichterbringung sind Ausgleichszahlungen fällig (siehe Gebührenordnung oben).\\\\
Für die \textbf{Pflege der Plätze} sind die nachstehenden Mannschaften verantwortlich:\\\\
Platz 1: Damen 30\\
Platz 2: Damen 40\\
Platz 3: Herren\\
Platz 4: Herren 40/50\\\\
Jede Mannschaft sollte mehrmals in der Saison vor oder nach dem Training auf den zugeordneten Plätzen Gras und Unkraut entfernen sowie die Plätze bis an den Rand sauber abziehen und Unebenheiten ausgleichen bzw. Schäden beheben.\\\\
Ein \textbf{Vereinsaustritt} muss vor dem 31. Dezember für das nächst folgende Jahr dem Vorstand schriftlich angezeigt werden.
\end{document}
